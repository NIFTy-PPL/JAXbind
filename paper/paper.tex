\documentclass[10pt,a4paper,onecolumn]{article}
\usepackage{marginnote}
\usepackage{graphicx}
\usepackage{xcolor}
\usepackage{authblk,etoolbox}
\usepackage{titlesec}
\usepackage{calc}
\usepackage{tikz}
\usepackage{hyperref}
\hypersetup{colorlinks,breaklinks=true,
            urlcolor=[rgb]{0.0, 0.5, 1.0},
            linkcolor=[rgb]{0.0, 0.5, 1.0}}
\usepackage{caption}
\usepackage{tcolorbox}
\usepackage{amssymb,amsmath}
\usepackage{ifxetex,ifluatex}
\usepackage{seqsplit}
\usepackage{xstring}

\usepackage{float}
\let\origfigure\figure
\let\endorigfigure\endfigure
\renewenvironment{figure}[1][2] {
    \expandafter\origfigure\expandafter[H]
} {
    \endorigfigure
}


\usepackage{fixltx2e} % provides \textsubscript
\usepackage[
  backend=biber,
%  style=alphabetic,
%  citestyle=numeric
]{biblatex}
\bibliography{paper.bib}

% --- Splitting \texttt --------------------------------------------------

\let\textttOrig=\texttt
\def\texttt#1{\expandafter\textttOrig{\seqsplit{#1}}}
\renewcommand{\seqinsert}{\ifmmode
  \allowbreak
  \else\penalty6000\hspace{0pt plus 0.02em}\fi}


% --- Pandoc does not distinguish between links like [foo](bar) and
% --- [foo](foo) -- a simplistic Markdown model.  However, this is
% --- wrong:  in links like [foo](foo) the text is the url, and must
% --- be split correspondingly.
% --- Here we detect links \href{foo}{foo}, and also links starting
% --- with https://doi.org, and use path-like splitting (but not
% --- escaping!) with these links.
% --- Another vile thing pandoc does is the different escaping of
% --- foo and bar.  This may confound our detection.
% --- This problem we do not try to solve at present, with the exception
% --- of doi-like urls, which we detect correctly.


\makeatletter
\let\href@Orig=\href
\def\href@Urllike#1#2{\href@Orig{#1}{\begingroup
    \def\Url@String{#2}\Url@FormatString
    \endgroup}}
\def\href@Notdoi#1#2{\def\tempa{#1}\def\tempb{#2}%
  \ifx\tempa\tempb\relax\href@Urllike{#1}{#2}\else
  \href@Orig{#1}{#2}\fi}
\def\href#1#2{%
  \IfBeginWith{#1}{https://doi.org}%
  {\href@Urllike{#1}{#2}}{\href@Notdoi{#1}{#2}}}
\makeatother

\newlength{\cslhangindent}
\setlength{\cslhangindent}{1.5em}
\newlength{\csllabelwidth}
\setlength{\csllabelwidth}{3em}
\newenvironment{CSLReferences}[3] % #1 hanging-ident, #2 entry spacing
 {% don't indent paragraphs
  \setlength{\parindent}{0pt}
  % turn on hanging indent if param 1 is 1
  \ifodd #1 \everypar{\setlength{\hangindent}{\cslhangindent}}\ignorespaces\fi
  % set entry spacing
  \ifnum #2 > 0
  \setlength{\parskip}{#2\baselineskip}
  \fi
 }%
 {}
\usepackage{calc}
\newcommand{\CSLBlock}[1]{#1\hfill\break}
\newcommand{\CSLLeftMargin}[1]{\parbox[t]{\csllabelwidth}{#1}}
\newcommand{\CSLRightInline}[1]{\parbox[t]{\linewidth - \csllabelwidth}{#1}}
\newcommand{\CSLIndent}[1]{\hspace{\cslhangindent}#1}

% --- Page layout -------------------------------------------------------------
\usepackage[top=3.5cm, bottom=3cm, right=2.54cm, left=2.54cm,
            headheight=2.2cm]{geometry}

% --- Default font ------------------------------------------------------------
\renewcommand\familydefault{\sfdefault}

% --- Style -------------------------------------------------------------------
\renewcommand{\bibfont}{\small \sffamily}
\renewcommand{\captionfont}{\small\sffamily}
\renewcommand{\captionlabelfont}{\bfseries}

% --- Section/SubSection/SubSubSection ----------------------------------------
\titleformat{\section}
  {\normalfont\sffamily\Large\bfseries}
  {}{0pt}{}
\titleformat{\subsection}
  {\normalfont\sffamily\large\bfseries}
  {}{0pt}{}
\titleformat{\subsubsection}
  {\normalfont\sffamily\bfseries}
  {}{0pt}{}
\titleformat*{\paragraph}
  {\sffamily\normalsize}


% --- Header / Footer ---------------------------------------------------------
\usepackage{fancyhdr}
\pagestyle{fancy}
\fancyhf{}
%\renewcommand{\headrulewidth}{0.50pt}
\renewcommand{\headrulewidth}{0pt}
\fancyhead[L]{}
\fancyhead[C]{}
\fancyhead[R]{}
\renewcommand{\footrulewidth}{0.0pt}

\fancyfoot[L]{}


\fancyfoot[R]{\sffamily \thepage}
\makeatletter
\let\ps@plain\ps@fancy
\fancyheadoffset[L]{4.5cm}
\fancyfootoffset[L]{4.5cm}

% --- Macros ---------

\definecolor{linky}{rgb}{0.0, 0.5, 1.0}

\newtcolorbox{repobox}
   {colback=red, colframe=red!75!black,
     boxrule=0.5pt, arc=2pt, left=6pt, right=6pt, top=3pt, bottom=3pt}

\newcommand{\ExternalLink}{%
   \tikz[x=1.2ex, y=1.2ex, baseline=-0.05ex]{%
       \begin{scope}[x=1ex, y=1ex]
           \clip (-0.1,-0.1)
               --++ (-0, 1.2)
               --++ (0.6, 0)
               --++ (0, -0.6)
               --++ (0.6, 0)
               --++ (0, -1);
           \path[draw,
               line width = 0.5,
               rounded corners=0.5]
               (0,0) rectangle (1,1);
       \end{scope}
       \path[draw, line width = 0.5] (0.5, 0.5)
           -- (1, 1);
       \path[draw, line width = 0.5] (0.6, 1)
           -- (1, 1) -- (1, 0.6);
       }
   }

% --- Title / Authors ---------------------------------------------------------
% patch \maketitle so that it doesn't center
\patchcmd{\@maketitle}{center}{flushleft}{}{}
\patchcmd{\@maketitle}{center}{flushleft}{}{}
% patch \maketitle so that the font size for the title is normal
\patchcmd{\@maketitle}{\LARGE}{\LARGE\sffamily}{}{}
% patch the patch by authblk so that the author block is flush left
\def\maketitle{{%
  \renewenvironment{tabular}[2][]
    {\begin{flushleft}}
    {\end{flushleft}}
  \AB@maketitle}}
\makeatletter
\renewcommand\AB@affilsepx{ \protect\Affilfont}
%\renewcommand\AB@affilnote[1]{{\bfseries #1}\hspace{2pt}}
\renewcommand\AB@affilnote[1]{{\bfseries #1}\hspace{3pt}}
\renewcommand{\affil}[2][]%
   {\newaffiltrue\let\AB@blk@and\AB@pand
      \if\relax#1\relax\def\AB@note{\AB@thenote}\else\def\AB@note{#1}%
        \setcounter{Maxaffil}{0}\fi
        \begingroup
        \let\href=\href@Orig
        \let\texttt=\textttOrig
        \let\protect\@unexpandable@protect
        \def\thanks{\protect\thanks}\def\footnote{\protect\footnote}%
        \@temptokena=\expandafter{\AB@authors}%
        {\def\\{\protect\\\protect\Affilfont}\xdef\AB@temp{#2}}%
         \xdef\AB@authors{\the\@temptokena\AB@las\AB@au@str
         \protect\\[\affilsep]\protect\Affilfont\AB@temp}%
         \gdef\AB@las{}\gdef\AB@au@str{}%
        {\def\\{, \ignorespaces}\xdef\AB@temp{#2}}%
        \@temptokena=\expandafter{\AB@affillist}%
        \xdef\AB@affillist{\the\@temptokena \AB@affilsep
          \AB@affilnote{\AB@note}\protect\Affilfont\AB@temp}%
      \endgroup
       \let\AB@affilsep\AB@affilsepx
}
\makeatother
\renewcommand\Authfont{\sffamily\bfseries}
\renewcommand\Affilfont{\sffamily\small\mdseries}
\setlength{\affilsep}{1em}


\ifnum 0\ifxetex 1\fi\ifluatex 1\fi=0 % if pdftex
  \usepackage[T1]{fontenc}
  \usepackage[utf8]{inputenc}

\else % if luatex or xelatex
  \ifxetex
    \usepackage{mathspec}
    \usepackage{fontspec}

  \else
    \usepackage{fontspec}
  \fi
  \defaultfontfeatures{Ligatures=TeX,Scale=MatchLowercase}

\fi
% use upquote if available, for straight quotes in verbatim environments
\IfFileExists{upquote.sty}{\usepackage{upquote}}{}
% use microtype if available
\IfFileExists{microtype.sty}{%
\usepackage{microtype}
\UseMicrotypeSet[protrusion]{basicmath} % disable protrusion for tt fonts
}{}

\usepackage{hyperref}
\hypersetup{unicode=true,
            pdftitle={JAXbind: Easy bindings to JAX},
            pdfborder={0 0 0},
            breaklinks=true}
\urlstyle{same}  % don't use monospace font for urls
\usepackage{color}
\usepackage{fancyvrb}
\newcommand{\VerbBar}{|}
\newcommand{\VERB}{\Verb[commandchars=\\\{\}]}
\DefineVerbatimEnvironment{Highlighting}{Verbatim}{commandchars=\\\{\}}
% Add ',fontsize=\small' for more characters per line
\newenvironment{Shaded}{}{}
\newcommand{\AlertTok}[1]{\textcolor[rgb]{1.00,0.00,0.00}{\textbf{#1}}}
\newcommand{\AnnotationTok}[1]{\textcolor[rgb]{0.38,0.63,0.69}{\textbf{\textit{#1}}}}
\newcommand{\AttributeTok}[1]{\textcolor[rgb]{0.49,0.56,0.16}{#1}}
\newcommand{\BaseNTok}[1]{\textcolor[rgb]{0.25,0.63,0.44}{#1}}
\newcommand{\BuiltInTok}[1]{#1}
\newcommand{\CharTok}[1]{\textcolor[rgb]{0.25,0.44,0.63}{#1}}
\newcommand{\CommentTok}[1]{\textcolor[rgb]{0.38,0.63,0.69}{\textit{#1}}}
\newcommand{\CommentVarTok}[1]{\textcolor[rgb]{0.38,0.63,0.69}{\textbf{\textit{#1}}}}
\newcommand{\ConstantTok}[1]{\textcolor[rgb]{0.53,0.00,0.00}{#1}}
\newcommand{\ControlFlowTok}[1]{\textcolor[rgb]{0.00,0.44,0.13}{\textbf{#1}}}
\newcommand{\DataTypeTok}[1]{\textcolor[rgb]{0.56,0.13,0.00}{#1}}
\newcommand{\DecValTok}[1]{\textcolor[rgb]{0.25,0.63,0.44}{#1}}
\newcommand{\DocumentationTok}[1]{\textcolor[rgb]{0.73,0.13,0.13}{\textit{#1}}}
\newcommand{\ErrorTok}[1]{\textcolor[rgb]{1.00,0.00,0.00}{\textbf{#1}}}
\newcommand{\ExtensionTok}[1]{#1}
\newcommand{\FloatTok}[1]{\textcolor[rgb]{0.25,0.63,0.44}{#1}}
\newcommand{\FunctionTok}[1]{\textcolor[rgb]{0.02,0.16,0.49}{#1}}
\newcommand{\ImportTok}[1]{#1}
\newcommand{\InformationTok}[1]{\textcolor[rgb]{0.38,0.63,0.69}{\textbf{\textit{#1}}}}
\newcommand{\KeywordTok}[1]{\textcolor[rgb]{0.00,0.44,0.13}{\textbf{#1}}}
\newcommand{\NormalTok}[1]{#1}
\newcommand{\OperatorTok}[1]{\textcolor[rgb]{0.40,0.40,0.40}{#1}}
\newcommand{\OtherTok}[1]{\textcolor[rgb]{0.00,0.44,0.13}{#1}}
\newcommand{\PreprocessorTok}[1]{\textcolor[rgb]{0.74,0.48,0.00}{#1}}
\newcommand{\RegionMarkerTok}[1]{#1}
\newcommand{\SpecialCharTok}[1]{\textcolor[rgb]{0.25,0.44,0.63}{#1}}
\newcommand{\SpecialStringTok}[1]{\textcolor[rgb]{0.73,0.40,0.53}{#1}}
\newcommand{\StringTok}[1]{\textcolor[rgb]{0.25,0.44,0.63}{#1}}
\newcommand{\VariableTok}[1]{\textcolor[rgb]{0.10,0.09,0.49}{#1}}
\newcommand{\VerbatimStringTok}[1]{\textcolor[rgb]{0.25,0.44,0.63}{#1}}
\newcommand{\WarningTok}[1]{\textcolor[rgb]{0.38,0.63,0.69}{\textbf{\textit{#1}}}}

% --- We redefined \texttt, but in sections and captions we want the
% --- old definition
\let\addcontentslineOrig=\addcontentsline
\def\addcontentsline#1#2#3{\bgroup
  \let\texttt=\textttOrig\addcontentslineOrig{#1}{#2}{#3}\egroup}
\let\markbothOrig\markboth
\def\markboth#1#2{\bgroup
  \let\texttt=\textttOrig\markbothOrig{#1}{#2}\egroup}
\let\markrightOrig\markright
\def\markright#1{\bgroup
  \let\texttt=\textttOrig\markrightOrig{#1}\egroup}


\IfFileExists{parskip.sty}{%
\usepackage{parskip}
}{% else
\setlength{\parindent}{0pt}
\setlength{\parskip}{6pt plus 2pt minus 1pt}
}
\setlength{\emergencystretch}{3em}  % prevent overfull lines
\providecommand{\tightlist}{%
  \setlength{\itemsep}{0pt}\setlength{\parskip}{0pt}}
\setcounter{secnumdepth}{0}
% Redefines (sub)paragraphs to behave more like sections
\ifx\paragraph\undefined\else
\let\oldparagraph\paragraph
\renewcommand{\paragraph}[1]{\oldparagraph{#1}\mbox{}}
\fi
\ifx\subparagraph\undefined\else
\let\oldsubparagraph\subparagraph
\renewcommand{\subparagraph}[1]{\oldsubparagraph{#1}\mbox{}}
\fi

\title{\texttt{JAXbind}: Easy bindings to JAX}

        \author[1, 2, 3*]{Jakob Roth}
          \author[1*]{Martin Reinecke}
          \author[1, 2, 4*]{Gordian Edenhofer}
    
      \affil[1]{Max Planck Institute for Astrophysics,
Karl-Schwarzschild-Str. 1, 85748 Garching, Germany}
      \affil[2]{Ludwig Maximilian University of Munich,
Geschwister-Scholl-Platz 1, 80539 Munich, Germany}
      \affil[3]{Technical University of Munich, Boltzmannstr. 3, 85748
Garching, Germany}
      \affil[4]{Department of Astrophysics, University of Vienna,
      T\"urkenschanzstr. 17, A-1180 Vienna, Austria}
      \affil[*]{These authors contributed equally.}
  \date{\vspace{-7ex}}

\begin{document}
\maketitle

\vspace{1em}

\hypertarget{summary}{%
\section{Summary}\label{summary}}

JAX is widely used in machine learning and scientific computing.
Scientific computing relies on existing high-performance code which we
would ideally like to use in JAX. Reimplementing the existing code in
JAX is often impractical and the existing interface in JAX for
connecting custom code requires deep knowledge of JAX and its C++
backend. The aim of \texttt{JAXbind} is to drastically lower the burden
of connecting custom functions implemented in other programming
languages to JAX. Specifically, \texttt{JAXbind} provides an easy-to-use
Python interface for defining custom, so-called JAX primitives
supporting any JAX transformations.

\hypertarget{statement-of-need}{%
\section{Statement of Need}\label{statement-of-need}}

The use of JAX (Bradbury et al., 2018) is widespread in the natural
sciences. JAX's powerful transformation system is of especially high
interest. It enables retrieving arbitrary derivatives of functions,
batch computations, and just-in-time code compilation for additional
performance. Its transformation system relies on all constituents of the
computation being written in JAX.

A plethora of high-performance code is not written in JAX and thus not
accessible from within JAX. Rewriting these is often infeasible and/or
inefficient. Ideally, we would like to intermix existing
high-performance code with JAX code. However, connecting code to JAX
requires knowledge of the internals of JAX and its C++ backend.

In this paper, we present \texttt{JAXbind}, a package for bridging any
function to JAX without in-depth knowledge of JAX's transformation
system. The interface is accessible from Python without requiring any
development in C++. The package is able to register any function, its
partial derivatives and their transpose functions as a JAX native call,
a so-called primitive. Derivatives, compilation rules, and batching
rules are automatically registered with JAX.

We believe \texttt{JAXbind} to be highly useful in scientific computing.
We intend to use this package to connect the Hartley transform and the
spherical harmonic transform from ducc (Reinecke, 2024) to NIFTy
(Edenhofer et al., 2024) as well as the radio interferometry response
from ducc with \texttt{resolve} (Arras et al., 2024) for radio
astronomy. Furthermore, we intend to connect the non-uniform FFT from
ducc with JAX for applications in strong-lensing astrophysics. We
envision many further applications within and outside of astrophysics.

To the best of our knowledge no other code currently exists for
connecting generic functions to JAX. The package that comes the closest
is Enzyme-JAX (EnzymeAD, 2024). Enzyme-JAX allows one to differentiate a
C++ function with Enzyme (W. Moses \& Churavy, 2020; W. S. Moses et al.,
2021, 2022) and connect it together with its derivative to JAX. However,
it enforces the use of Enzyme for deriving derivatives and does not
allow for connecting arbitrary code to JAX.

\hypertarget{automatic-differentiation-and-code-example}{%
\section{Automatic Differentiation and Code
Example}\label{automatic-differentiation-and-code-example}}

Automatic differentiation is a core feature of JAX and often one of the
main reasons for using it. Thus, it is essential that custom functions
registered with JAX support automatic differentiation. In the following,
we will outline which functions our package respectively JAX requires to
enable automatic differentiation. For simplicity, we assume that we want
to connect the nonlinear function \(f(x_1,x_2) = x_1x_2^2\) to JAX. The
\texttt{JAXbind} package expects the Python function for \(f\) to take
three positional arguments. The first argument, \texttt{out}, is a
\texttt{tuple} into which the function results are written. The second
argument is also a \texttt{tuple} containing the input to the function,
in our case, \(x_1\) and \(x_2\). Via \texttt{kwargs\_dump}, potential
keyword arguments given to the later registered Jax primitive can be
forwarded to \texttt{f} in serialized form.

\begin{Shaded}
\begin{Highlighting}[]
\ImportTok{import}\NormalTok{ jaxbind}

\KeywordTok{def}\NormalTok{ f(out, args, kwargs\_dump):}
\NormalTok{    kwargs }\OperatorTok{=}\NormalTok{ jaxbind.load\_kwargs(kwargs\_dump)}
\NormalTok{    x1, x2 }\OperatorTok{=}\NormalTok{ args}
\NormalTok{    out[}\DecValTok{0}\NormalTok{][()] }\OperatorTok{=}\NormalTok{ x1 }\OperatorTok{*}\NormalTok{ x2}\OperatorTok{**}\DecValTok{2}
\end{Highlighting}
\end{Shaded}

JAX's automatic differentiation engine can compute the Jacobian-vector
product \texttt{jvp} and vector-Jacobian product \texttt{vjp} of JAX
primitives. The Jacobian-vector product in JAX is a function applying
the Jacobian of \(f\) at a position \(x\) to a tangent vector. In
mathematical nomenclature this operation is called the pushforward of
\(f\) and can be denoted as \(\partial f(x): T_x X \mapsto T_{f(x)} Y\),
with \(T_x X\) and \(T_{f(x)} Y\) being the tangent spaces of \(X\) and
\(Y\) at the positions \(x\) and \(f(x)\). As the implementation of
\(f\) is not JAX native, JAX cannot automatically compute the
\texttt{jvp}. Instead, an implementation of the pushforward has to be
provided, which \texttt{JAXbind} will register as the \texttt{jvp} of
the JAX primitive of \(f\). For our example, this
Jacobian-vector-product function is given by
\(\partial f(x_1,x_2)(dx_1,dx_2) = x_2^2dx_1 + 2x_1x_2dx_2\).

\begin{Shaded}
\begin{Highlighting}[]
\KeywordTok{def}\NormalTok{ f\_jvp(out, args, kwargs\_dump):}
\NormalTok{    kwargs }\OperatorTok{=}\NormalTok{ jaxbind.load\_kwargs(kwargs\_dump)}
\NormalTok{    x1, x2, dx1, dx2 }\OperatorTok{=}\NormalTok{ args}
\NormalTok{    out[}\DecValTok{0}\NormalTok{][()] }\OperatorTok{=}\NormalTok{ x2}\OperatorTok{**}\DecValTok{2} \OperatorTok{*}\NormalTok{ dx1 }\OperatorTok{+} \DecValTok{2} \OperatorTok{*}\NormalTok{ x1 }\OperatorTok{*}\NormalTok{ x2 }\OperatorTok{*}\NormalTok{ dx2}
\end{Highlighting}
\end{Shaded}

The vector-Jacobian product \texttt{vjp} in JAX is the linear transpose
of the Jacobian-vector product. In mathematical nomenclature this is the
pullback \((\partial f(x))^{T}: T_{f(x)}Y \mapsto T_x X\) of \(f\).
Analogously to the \texttt{jvp}, the user has to implement this function
as JAX cannot automatically construct it. For our example function, the
vector-Jacobian product is
\((\partial f(x_1,x_2))^{T}(dy) = (x_2^2dy, 2x_1x_2dy)\).

\begin{Shaded}
\begin{Highlighting}[]
\KeywordTok{def}\NormalTok{ f\_vjp(out, args, kwargs\_dump):}
\NormalTok{    kwargs }\OperatorTok{=}\NormalTok{ jaxbind.load\_kwargs(kwargs\_dump)}
\NormalTok{    x1, x2, dy }\OperatorTok{=}\NormalTok{ args}
\NormalTok{    out[}\DecValTok{0}\NormalTok{][()] }\OperatorTok{=}\NormalTok{ x2}\OperatorTok{**}\DecValTok{2} \OperatorTok{*}\NormalTok{ dy}
\NormalTok{    out[}\DecValTok{1}\NormalTok{][()] }\OperatorTok{=} \DecValTok{2} \OperatorTok{*}\NormalTok{ x1 }\OperatorTok{*}\NormalTok{ x2 }\OperatorTok{*}\NormalTok{ dy}
\end{Highlighting}
\end{Shaded}

To just-in-time compile the function, JAX needs to abstractly evaluate
the code, i.e.~it needs to be able to know the shape and dtype of the
output of the custom function given only the shape and dtype of the
input. We have to provide these abstract evaluation functions returning
the output shape and dtype given an input shape and dtype for \texttt{f}
as well as for the \texttt{vjp} application. The output shape of the
\texttt{jvp} is identical to the output shape of \texttt{f} itself and
does not need to be specified again.

\begin{Shaded}
\begin{Highlighting}[]
\KeywordTok{def}\NormalTok{ f\_abstract(}\OperatorTok{*}\NormalTok{args, }\OperatorTok{**}\NormalTok{kwargs):}
    \ControlFlowTok{assert}\NormalTok{ args[}\DecValTok{0}\NormalTok{].shape }\OperatorTok{==}\NormalTok{ args[}\DecValTok{1}\NormalTok{].shape}
    \ControlFlowTok{return}\NormalTok{ ((args[}\DecValTok{0}\NormalTok{].shape, args[}\DecValTok{0}\NormalTok{].dtype),)}

\KeywordTok{def}\NormalTok{ f\_abstract\_T(}\OperatorTok{*}\NormalTok{args, }\OperatorTok{**}\NormalTok{kwargs):}
    \ControlFlowTok{return}\NormalTok{ (}
\NormalTok{        (args[}\DecValTok{0}\NormalTok{].shape, args[}\DecValTok{0}\NormalTok{].dtype),}
\NormalTok{        (args[}\DecValTok{0}\NormalTok{].shape, args[}\DecValTok{0}\NormalTok{].dtype),}
\NormalTok{    )}
\end{Highlighting}
\end{Shaded}

We have now defined all ingredients necessary to register a JAX
primitive for our function \(f\) using the \texttt{JAXbind} package.

\begin{Shaded}
\begin{Highlighting}[]
\NormalTok{f\_jax }\OperatorTok{=}\NormalTok{ jaxbind.get\_nonlinear\_call(}
\NormalTok{    f, (f\_jvp, f\_vjp), f\_abstract, f\_abstract\_T}
\NormalTok{)}
\end{Highlighting}
\end{Shaded}

\texttt{f\_jax} is a JAX primitive registered via the \texttt{JAXbind}
package supporting all JAX transformations. We can now compute the
\texttt{jvp} and \texttt{vjp} of the new JAX primitive and even
jit-compile and batch it.

\begin{Shaded}
\begin{Highlighting}[]
\ImportTok{import}\NormalTok{ jax}
\ImportTok{import}\NormalTok{ jax.numpy }\ImportTok{as}\NormalTok{ jnp}

\NormalTok{inp }\OperatorTok{=}\NormalTok{ (jnp.full((}\DecValTok{4}\NormalTok{,}\DecValTok{3}\NormalTok{), }\FloatTok{4.}\NormalTok{), jnp.full((}\DecValTok{4}\NormalTok{,}\DecValTok{3}\NormalTok{), }\FloatTok{2.}\NormalTok{))}
\NormalTok{tan }\OperatorTok{=}\NormalTok{ (jnp.full((}\DecValTok{4}\NormalTok{,}\DecValTok{3}\NormalTok{), }\FloatTok{1.}\NormalTok{), jnp.full((}\DecValTok{4}\NormalTok{,}\DecValTok{3}\NormalTok{), }\FloatTok{1.}\NormalTok{))}
\NormalTok{res, res\_tan }\OperatorTok{=}\NormalTok{ jax.jvp(f\_jax, inp, tan)}

\NormalTok{cotan }\OperatorTok{=}\NormalTok{ (jnp.full((}\DecValTok{4}\NormalTok{,}\DecValTok{3}\NormalTok{), }\FloatTok{6.}\NormalTok{),)}
\NormalTok{res, f\_vjp }\OperatorTok{=}\NormalTok{ jax.vjp(f\_jax, }\OperatorTok{*}\NormalTok{inp)}
\NormalTok{res\_cotan }\OperatorTok{=}\NormalTok{ f\_vjp(cotan)}

\NormalTok{f\_jax\_jit }\OperatorTok{=}\NormalTok{ jax.jit(f\_jax)}
\NormalTok{res }\OperatorTok{=}\NormalTok{ f\_jax\_jit(}\OperatorTok{*}\NormalTok{inp)}
\end{Highlighting}
\end{Shaded}

\hypertarget{higher-order-derivatives-and-linear-functions}{%
\section{Higher Order Derivatives and Linear
Functions}\label{higher-order-derivatives-and-linear-functions}}

JAX supports higher order derivatives and can differentiate a
\texttt{jvp} or \texttt{vjp} with respect to the position at which the
Jacobian was taken. Similar to first derivatives, JAX can not
automatically compute higher derivatives of a general function \(f\)
that is not natively implemented in JAX. Higher order derivatives would
again need to be provided by the user. For many algorithms, first
derivatives are sufficient, and higher order derivatives are often not
implemented by the high-performance codes. Therefore, the current
interface of \texttt{JAXbind} is, for simplicity, restricted to first
derivatives. In the future, the interface could be easily expanded if
specific use cases require higher order derivatives.

In scientific computing, linear functions such as, e.g., spherical
harmonic transforms are widespread. If the function \(f\) is linear,
differentiation becomes trivial. Specifically for a linear function
\(f\), the pushforward respectively the \texttt{jvp} of \(f\) is
identical to \(f\) itself and independent of the position at which it is
computed. Expressed in formulas, \(\partial f(x)(dx) = f(dx)\) if \(f\)
is linear in \(x\). Analogously, the pullback respectively the
\texttt{vjp} becomes independent of the initial position and is given by
the linear transpose of \(f\), thus
\((\partial f(x))^{T}(dy) = f^T(dy)\). Also, all higher order
derivatives can be expressed in terms of \(f\) and its transpose. To
make use of these simplifications, \texttt{JAXbind} provides a special
interface for linear functions, supporting higher order derivatives,
only requiring an implementation of the function and its transpose.

\hypertarget{platforms}{%
\section{Platforms}\label{platforms}}

Currently, \texttt{JAXbind} only has CPU but no GPU support. With some
expertise on Python bindings for GPU kernels adding GPU support should
be fairly simple. We especially want to highlight that the interfacing
with the JAX automatic differentiation engine is identical for CPU and
GPU.

\hypertarget{acknowledgements}{%
\section{Acknowledgements}\label{acknowledgements}}

We would like to thank Dan Foreman-Mackey for his detailed guide
(https://dfm.io/posts/extending-jax/) on connecting C++ code to JAX.
Jakob Roth acknowledges financial support from the German Federal
Ministry of Education and Research (BMBF) under grant 05A23WO1
(Verbundprojekt D-MeerKAT III). Gordian Edenhofer acknowledges support
from the German Academic Scholarship Foundation in the form of a PhD
scholarship (``Promotionsstipendium der Studienstiftung des Deutschen
Volkes'').

\hypertarget{references}{%
\section{References}\label{references}}

\hypertarget{refs}{}
\begin{CSLReferences}{1}{0}
\leavevmode\hypertarget{ref-Resolve2024}{}%
Arras, P., Roth, J., Ding, S., Reinecke, M., Fuchs, R., \& Johnson, V.
(2024). \emph{RESOLVE}. \url{https://gitlab.mpcdf.mpg.de/ift/resolve}

\leavevmode\hypertarget{ref-Jax2018}{}%
Bradbury, J., Frostig, R., Hawkins, P., Johnson, M. J., Leary, C.,
Maclaurin, D., Necula, G., Paszke, A., VanderPlas, J., Wanderman-Milne,
S., \& Zhang, Q. (2018). \emph{{JAX}: Composable transformations of
{P}ython+{N}um{P}y programs} (Version 0.3.13) {[}Computer software{]}.
\url{http://github.com/google/jax}

\leavevmode\hypertarget{ref-Edenhofer2023NIFTyRE}{}%
Edenhofer, G., Frank, P., Roth, J., Leike, R. H., Guerdi, M.,
Scheel-Platz, L. I., Guardiani, M., Eberle, V., Westerkamp, M., \&
Enßlin, T. A. (2024). \emph{{Re-Envisioning Numerical Information Field
Theory (NIFTy.re): A Library for Gaussian Processes and Variational
Inference}}. \url{http://arxiv.org/abs/2402.16683}

\leavevmode\hypertarget{ref-Moses2024}{}%
EnzymeAD. (2024). \emph{{Enzyme-JAX}} (Version 0.0.6) {[}Computer
software{]}. \url{https://github.com/EnzymeAD/Enzyme-JAX}

\leavevmode\hypertarget{ref-Moses2020}{}%
Moses, W., \& Churavy, V. (2020). Instead of rewriting foreign code for
machine learning, automatically synthesize fast gradients. In H.
Larochelle, M. Ranzato, R. Hadsell, M. F. Balcan, \& H. Lin (Eds.),
\emph{Advances in neural information processing systems} (Vol. 33, pp.
12472--12485). Curran Associates, Inc.
\url{https://proceedings.neurips.cc/paper/2020/file/9332c513ef44b682e9347822c2e457ac-Paper.pdf}

\leavevmode\hypertarget{ref-Moses2021}{}%
Moses, W. S., Churavy, V., Paehler, L., Hückelheim, J., Narayanan, S. H.
K., Schanen, M., \& Doerfert, J. (2021). Reverse-mode automatic
differentiation and optimization of GPU kernels via enzyme.
\emph{Proceedings of the International Conference for High Performance
Computing, Networking, Storage and Analysis}.
\url{https://doi.org/10.1145/3458817.3476165}

\leavevmode\hypertarget{ref-Moses2022}{}%
Moses, W. S., Narayanan, S. H. K., Paehler, L., Churavy, V., Schanen,
M., Hückelheim, J., Doerfert, J., \& Hovland, P. (2022). Scalable
automatic differentiation of multiple parallel paradigms through
compiler augmentation. \emph{Proceedings of the International Conference
on High Performance Computing, Networking, Storage and Analysis}.
ISBN:~\href{https://worldcat.org/isbn/9784665454445}{9784665454445}

\leavevmode\hypertarget{ref-ducc0}{}%
Reinecke, M. (2024). \emph{{DUCC}: Distinctly useful code collection}
(Version 0.33.0) {[}Computer software{]}.
\url{https://gitlab.mpcdf.mpg.de/mtr/ducc}

\end{CSLReferences}

\end{document}
